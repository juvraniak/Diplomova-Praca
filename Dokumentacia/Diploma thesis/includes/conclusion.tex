\indent Cieľom práce bolo zanalyzovať populárne konzolové rozhrania ako aj skriptovacie jazyky, ktoré sú často využívané pri administrácii počítačových systémov. Rovnako bolo potrebné identifikovať výhody ako aj nedostatky jednotlivých riešení, zhodnotiť ich a nájsť medzi nimi rozumný prienik. 
\newline
\indent Návrhu aplikácie predchádzalo štúdium problematiky prekladu jazykov. Vedomosti nadobudnuté z analýzy sme sa následne snažili využiť pri navrhovaní a implementovaní aplikácie. 
\newline
\indent Pri návrhu aplikácie sme dbali na to, aby jazyk, ktorý aplikácia poskytuje, bol čo najzrozumiteľnejší pre používateľov. Rovnako sme uskutočnili dôkladnú špecifikáciu prípadov použitia, ktoré pri programovaní značne zjednodušili celý proces vývoja. Pred samotnou implementáciou sme rozhodli, ktoré návrhové vzory použijeme v práci, a pomocu nich sme vytvorili prvé funkčné demo aplikácie. 
\newline
\indent Následne sme implementovali funkcionalitu stanovenú v návrhovej časti. Pre úspešné vytvorenie aplikácie boli potrebné zásahy do prvotného návrhu. Avšak v konečnom dôsledku sa nám podarilo naimplementovať aplikáciu, ktorá podporuje skriptovací, ako aj interaktívny mód. Taktiež sme vytvorili spôsob integrácie s inými aplikáciami. Naimplementovali sme základné štruktúry jazyka, ako funkcia for cyklus, if podmienka. Taktiež podporujeme a rozlišujeme medzi lokálnymi a globálnymi scopami. Na základe zhodnotenia výsledkov je navrhnutá aplikácia o čosi pomalšia ako ostatné jazyky, avšak jej prednosťou je skutočnosť, že je multiplatformová a má rovnaký výkon na každej platforme.
\newline
\indent Na základe spomenutých informácií, je možné povedať, že cieľ práce bol splnený a výsledná práca je dostačujúca.