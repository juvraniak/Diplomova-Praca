\indent  Diplomová práca sa zameriava na analýzu existujúcich skriptovacích jazykov a návrh nového univerzálneho konzolového rozhrania, ktoré je orientované na administrátorské úlohy. Výsledkom a zároveň cieľom práce je vytvoriť plne funkčné administrátorské rozhranie, ktoré bude jednoducho rozšíriteľné pomocou pluginov, umožňujúce pracovať v interaktívnom, ako aj skriptovacom móduse, na rôznych operačných systémoch podporovaných \acrshort{jvm}.
\newline
\indent  V prvej časti práce vychádzame z teoretických poznatkov, charakterizujeme operačné systémy, ich rôznorodosť a softvérové vybavenie. Absencia jednotnej platformy na vytváranie skrípt, vo väčšine prípadov, vyžaduje duplikovanie alebo viacnásobnú implementáciu skriptu pre konkrétny operačný systém. Čiastočným riešením uvedeného problému je použitie skriptovacieho jazyka s podporou cieľových platform. Zásadným nedostatkom skriptovacích jazykov pri riešení problému je nedostatok syntaktických a funkčných konštrukcií, ktoré sú overené časom a široko používané, ako napríklad pajpa alebo presmerovanie štandardného a chybového výstupu. Okrem skriptovacích jazykov, sme otestovali emulátory, ktoré zabezpečujú preklad platformovo špecifického jazyka do jazyka spustiteľného na konkrétnej platforme. Ďalšia časť teoretickej analýzy podrobne charakterizuje preklad jazykov, kde sú opísané postupy fungovania kompilátorov programovacích jazykov. V nadväznosti na spracovanú problematiku sme špecifikovali prípady použitia aplikácie, určili programovací jazyk pre úspešné vypracovanie zadania, vymedzili návrhové vzory, ktoré nám umožnili sprehľadniť zdrojový kód a na záver sme podrobne opísali návrh aplikácie spolu s prvotným návrhom tried. 
\newline
\indent Výsledkom diplomovej práce je administrátorské rozhranie, rozšíriteľné pomocou pluginov, ktoré umožňuje pracovať v interaktívnom, ako aj skriptovacom móduse, na rôznych operačných systémoch podporovaných \acrshort{jvm}. Okrem uvedeného sú výstupom práce aj možnosti a predpoklady integrácie s ďalšími enterprise nástrojmi, a v neposledom rade poskytneme knižnicu obsahujúcu rozhrania, pomocou ktorých sa dá funkcionalita systému jednoducho rozšíriť.