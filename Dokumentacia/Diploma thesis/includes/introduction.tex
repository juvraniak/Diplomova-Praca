\indent  Diplomová práca sa zameriava na analýzu existujúcich skriptovacích jazykov a návrh nového univerzálneho konzolového rozhrania, ktoré je  zamerané na adminstrátorské úlohy. Výsledkom a zároveň cieľom práce je vytvoriť plne funkčné administrátorské rozhranie, ktoré bude jednoducho rozšíriteľné pomocou pluginov, umožňuje pracovať v interaktívnom, ako aj skriptovacom móduse, na rôznych operačných systémoch podporovaných \acrshort{jvm}.
\newline
\indent  V prvej časti práce sme sa zamerali na rôznorodosť operačných systémov a absencia jednotnej platformy na vytváranie skrípt vo väčšine prípadov vyžadujú ich duplikovanie alebo viacnásobnú implementáciu. Čiastočným riešením tohto problému je použitie skriptovacieho jazyka s podporou cieľových platform. Zásadným problémom skriptovacích jazykov pri riešení tohto problému je absencia syntaktických a funkčných konštrukcií, ktoré sú už overené a široko používané, ako napríklad pajpa alebo presmerovanie štandardného a chybového vstupu a výstupu. Okrem skriptovacách jazykov sme si vyskúšali emulátory, ktoré zabezpečujú preklad platformovo špecifického jazyka do jazyka spustiteľného na konkrétnej platforme. Ďalšia časť teoretickej analýzy sa venujeme prekladu jazykov, kde sú popísané, postupy ako fungujú kompilítory programovacích jazykov.  Následne sa venujeme návrhu celej práce, kde sme špecifikovali prípady použitia aplikácie, vybrali programovací jazyk pre úspešné vypracovanie zadania, popísali návrhové vzory, ktoré nám umožnili sprehľadniť zdrojový kód a na záver špecifikovali návrh aplikácie spolu s prvotným návrhom tried. 
\newline
\indent Ako bolo spomenuté výsledkom práce je plne funkčné administrátorské rozhranie, ktoré je jednoducho rozšíriteľné pomocou pluginov, umožňuje pracovať v interaktívnom, ako aj skriptovacom móduse, na rôznych operačných systémoch podporovaných \acrshort{jvm}. Taktiež poskytuje možnosti integrácie s ďalšími enterprise nástrojmi a v neposledom rade poskytuje knižnicu obsahujúcu rozhrania pomocou, ktorých sa dá funkcionalita systému jednoducho rozšíriť.