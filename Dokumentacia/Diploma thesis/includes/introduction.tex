\indent V dnešnej dobe rôznorodosť operačných systémov a absencia jednotnej platformy na vytváranie skrípt vo väčšine prípadov vyžadujú ich duplikovanie alebo viacnásobnú implementáciu. Čiastočným riešením tohto problému je použitie skriptovacieho jazyka s podporou cieľových platform. Zásadným problémom skriptovacích jazykov pri riešení tohto problému je absencia syntaktických a funkčných konštrukcií, ktoré sú už overené a široko používané, ako napríklad pajpa, izolovanie príkazov alebo presmerovanie štandardného a chybového vstupu a výstupu. Cieľom práce je analyzovať populárne konzolové rozhrania (napr. Bourne Shell, Power Shell, C-Shell) a skriptovacie jazyky (napr. Python, Groovy, Lua), porovnať ich syntax, funkcionality a limity. Následne navrhnúť nové konzolové rozhranie, ktoré bude spájať funkcionality identifikované ako výhody počas analýzy so zameraním na administrátorské úlohy. Pri implementácií je tiež kľúčovým faktorom identifikácia nových funkcionalít, ktoré by mohli uľahčiť vývoj robustných skrípt. Rozhranie musí umožňovať interaktívny aj skriptovaný módus. Očakáva sa možnosť integrácie rozhrania do iných systémov rôznej veľkosti a komplexity, od malých utilít a rutín až po enterprise aplikácie a ľahká rozšíriteľnosť rozhrania o nové príkazy a funkcionality.